\documentclass{resume} % Use the custom resume.cls style

\usepackage[left=0.4 in,top=0.4in,right=0.4 in,bottom=0.4in]{geometry}
\usepackage{tgheros}
\renewcommand{\familydefault}{\sfdefault}
\newcommand{\tab}[1]{\hspace{.2667\textwidth}\rlap{#1}} 
\newcommand{\itab}[1]{\hspace{0em}\rlap{#1}}
\name{Nome Sobrenome} % Your name
\address{+55(XX) XXXXX-XXXX \\ Cidade, UF} 
\address{\href{mailto:contact@mail.com}{contact@mail.com} \\ \href{https://linkedin.com/linkedin}{linkedin.com/linkedin} \\ \href{www.site.com}{www.site.com}}  %

\begin{document}

%----------------------------------------------------------------------------------------
%	SUMMARY
%----------------------------------------------------------------------------------------

\begin{rSection}{Resumo}

{Lorem ipsum dolor sit amet, consectetur adipiscing elit. Praesent molestie eros lectus, quis vehicula sapien lacinia eget. Proin quis tortor eu felis dictum auctor. Praesent id purus congue, commodo massa ut, tincidunt enim. Quisque massa lacus, eleifend sit amet ullamcorper nec, placerat ut libero. Mauris facilisis mattis eros, vitae fringilla mauris fringilla vel. Etiam nulla ex, feugiat id euismod in, porta vel orci. Suspendisse sit amet rhoncus purus. Vestibulum odio justo, blandit vehicula dignissim et, posuere et nulla. Phasellus velit lorem, ultrices id lacinia condimentum, imperdiet non risus. Suspendisse porta laoreet leo vel egestas. Maecenas at mi sit amet nunc fermentum ultrices. Mauris ex augue, interdum eget sem eu, pulvinar auctor velit. Praesent id risus a nunc aliquet suscipit.}


\end{rSection}

%----------------------------------------------------------------------------------------
%   EXPERIENCE
%----------------------------------------------------------------------------------------

\begin{rSection}{Experiência}

{\bf Desenvolvedor Java Junior \hfill {jan 2024 - atual}}\\
{Empresa – {Cidade, UF}}\\
- Fiz A que melhorou a eficiência em X\%\\
- Implementei B\\
- Alcancei C utilizando X, Y e Z\\

\vspace{1.25em}

{\bf Estagiário em Desenvolvimento de Software \hfill {jan 2023 - jan 2024}}\\
{Empresa – {Cidade, UF}}\\
- Fiz A que melhorou a eficiência em X\%\\
- Implementei B\\
- Alcancei C utilizando X, Y e Z\\

\end{rSection} 

%----------------------------------------------------------------------------------------
%	EDUCATION
%----------------------------------------------------------------------------------------

\begin{rSection}{Educação}

{\bf Universidade tal tal tal – UTTT \hfill {jan 2022 - jan 2025}}\\
{Bacharelado em Ciência da Computação}\\
- Relevant Coursework: A, B, C, and D.\\
- Relevant Coursework: A, B, C, and D.\\
- Relevant Coursework: A, B, C, and D.\\

\vspace{1.25em}

{\bf Universidade tal tal tal - UTTT \hfill {jan 2022 - jan 2025}}\\
{Técnico em Informática}\\
- Relevant Coursework: A, B, C, and D.\\
- Relevant Coursework: A, B, C, and D.\\
- Relevant Coursework: A, B, C, and D.\\

\end{rSection}

%----------------------------------------------------------------------------------------
%	SKILLS
%----------------------------------------------------------------------------------------

\begin{rSection}{Habilidades}

\begin{tabular}{ @{} >{\bfseries}l @{\hspace{6ex}} l }
Linguagens: &  Java, JavaScript, TypeScript, C\#\\
Front-end: & Next.Js, React, Angular, Tailwind CSS\\
Back-end: & Spring Boot, Spring Data JPA, Spring Webflux, Spring Security, Hibernate\\
Banco de dados: & MySQL, PostgreSQL, Oracle, SQLite, H2, MongoDB, Redis\\
Mobile: & Ionic/Angular, React Native\\
DevOps/Cloud: & Docker, Git, GitHub, Maven, Gradle, AWS\\
Mensageria: & RabbitMQ, Apache Kafka\\
Ferramentas: & JUnit, Mockito, Flyway, Liquibase, Swagger, Linux\\
\end{tabular}\\

\end{rSection}

%----------------------------------------------------------------------------------------
%	LANGUAGES
%----------------------------------------------------------------------------------------

\vspace{4em}

\begin{rSection}{IDIOMAS}

{- Inglês avançado (C1)}\\
{- Português nativo }\\

\end{rSection}

%----------------------------------------------------------------------------------------
%	RELEVANT CERTIFICATES
%----------------------------------------------------------------------------------------

\begin{rSection}{Certificados Relevantes}

{Lorem ipsum dolor sit amet, consectetur adipiscing elit. Praesent molestie eros lectus, quis vehicula sapien lacinia eget. Proin quis tortor eu felis dictum auctor. Praesent id purus congue, commodo massa ut, tincidunt enim. Quisque massa lacus, eleifend sit amet ullamcorper nec, placerat ut libero.}\\

\end{rSection}

%----------------------------------------------------------------------------------------
%   PROJECTS
%----------------------------------------------------------------------------------------

\begin{rSection}{Projetos}
\vspace{-1.25em}

\item \textbf{1 – Projeto X}\\
\vspace{0.25em}
{\textbf {Visão geral:} Lorem ipsum dolor sit amet, consectetur adipiscing elit. Praesent molestie eros lectus.}\\
{\textbf {Tecnologias:} Java, Spring Boot 3, Hibernate, MySQL, Flyway, Lombok, JUnit, Mockito, Swagger, JWT}\\
{\textbf {Repositório: } \href{https://github.com/}{https://github.com/}}\\

\vspace{1.25em}

\item \textbf{2 – Projeto Y}\\
\vspace{0.25em}
{\textbf {Visão geral:} Mauris facilisis mattis eros, vitae fringilla mauris fringilla vel. Etiam nulla ex, feugiat id euismo.}\\
{\textbf {Tecnologias:}  Next.Js, TypeScript, React, Swiper, CSS, Vercel, Node.Js, ESLint, Prettier}\\
{\textbf {Repositório: } \href{https://github.com/}{https://github.com/}}\\

\vspace{1.25em}

\item \textbf{3 – Projeto Z}\\
\vspace{0.25em}
{\textbf {Visão geral:} Phasellus velit lorem, ultrices id lacinia condimentum, imperdiet non risus.}\\
{\textbf {Tecnologias:} : Node.js, Fastify, Prisma, SQLite, Swagger, React, Vite, Tailwind CSS, Axios}\\
{\textbf {Repositório: } \href{https://github.com/}{https://github.com/}}\\
\end{rSection} 

%----------------------------------------------------------------------------------------

\end{document}
